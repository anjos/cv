\typeout{ ====================================================================}
\typeout{ This is file snsf-publist, created at Tue 22 Jan 2013 09:05:58 CET  }
\typeout{ Andre Anjos <andre.anjos@idiap.ch> }
\typeout{ ====================================================================}

\documentclass[11pt,a4paper,sans]{moderncv}
\usepackage[T1]{fontenc}

% styles: 'casual' (default), 'classic', 'oldstyle','banking'
\moderncvstyle{casual}

% colors: 'blue' (default), 'orange', 'green', 'red', 'purple', 'grey', 'black'
\moderncvcolor{blue}

% sorting=none will preserve the bib file order
\usepackage{csquotes}
\usepackage[backend=biber,maxbibnames=99,bibstyle=publist,plauthorhandling=highlight,plnumbered=true,marginyear=true,sorting=ddnt]{biblatex}
\setlength\bibitemsep{1.5\itemsep}
\plauthorname[André]{Anjos}

\addbibresource{filtered.bib}

% adjust the page margins
\usepackage[scale=0.8]{geometry}
% if you want to change the width of the column with the dates
%\setlength{\hintscolumnwidth}{3cm}

% for the 'classic' style, if you want to force the width allocated to your
% name and avoid line breaks. be careful though, the length is normally
% calculated to avoid any overlap with your personal info; use this at your own
% typographical risks...

%\setlength{\makecvtitlenamewidth}{10cm}

% stores the starting year (current year - 4)
\usepackage{calc}
\newcounter{startyear}
\newcounter{currentyear}

% hyperref setup
\definecolor{linkcolor}{rgb}{0.1,0.1,0.4}

\AfterPreamble{\hypersetup{
  pdfauthor={Andre Anjos},
  pdftitle={Publication list for Andre Anjos (last 5 years)},
  pdfsubject={Publication list for for Andre Anjos (last 5 years)},
  urlcolor=blue,
  colorlinks=true,    % false: boxed links; true: colored links
  linkcolor=linkcolor, % color of internal links (change box color with linkbordercolor)
  citecolor=linkcolor, % color of links to bibliography
  filecolor=linkcolor, % color of file links
  urlcolor=linkcolor   % color of external links
}}

\DeclareFieldFormat{labelnumberwidth}{}
\setlength{\biblabelsep}{0pt}

% personal data
\firstname{André}
\familyname{Anjos}

% optional, remove / comment the line if not wanted
\title{Signal Processing Engineer, D.Sc.}

% optional, remove / comment the line if not wanted
\address{Idiap Research Institute, rue Marconi 19, Centre du Parc, Martigny,
Switzerland, CH-1920}{}

% optional, remove / comment the line if not wanted
\mobile{+41767092708}

% optional, remove / comment the line if not wanted
\phone{+41277217763}

% optional, remove / comment the line if not wanted
\email{andre.anjos@idiap.ch}

% optional, remove / comment the line if not wanted
\homepage{anjos.ai}

% optional, remove / comment the line if not wanted
%\extrainfo{}

% optional, remove / comment the line if not wanted
% '64pt' is the height the picture must be resized to, 0.4pt is the thickness
% of the frame around it (put it to 0pt for no frame) and 'picture' is the name
% of the picture file
\photo[64pt][0.4pt]{picture}

% optional, remove / comment the line if not wanted
%\quote{Some quote}

\begin{document}

\setcounter{startyear}{\year-6}
\setcounter{currentyear}{\the\year-1}
\section{Research Output for André Anjos (\thestartyear\ -- \thecurrentyear)}

\cvitem{Complete list}{\url{https://anjos.ai}}
\cvitem{H-index}{ = 26, 19900+ citations}
\cvitem{Scholar}{\url{http://scholar.google.com/citations?hl=en&user=pAfLhMoAAAAJ}}

% for numerical labels: \renewcommand{\bibliographyitemlabel}{\@biblabel{\arabic{enumiv}}}
\renewcommand{\refname}{Research Output List (reverse chronological order)}
\nocite{*}

\printbibliography[title={Peer Reviewed Journal Articles}, type=article]

\printbibliography[title={Peer Reviewed Books}, type=incollection]

\printbibliography[title={Peer Reviewed Conference Proceedings}, type=inproceedings]

\printbibliography[title={Patents}, type=patent]

\printbibliography[title={Open Access Archive}, type=misc]

\section{Open Data}

\begin{description}
  \item[2017] \textit{COHFACE} - A dataset for remote photoplethysmography
    using RGB data in realistic conditions;
    URL: \url{https://www.idiap.ch/dataset/cohface}
\end{description}

\section{Open Software}

\begin{description}

  \item[\thecurrentyear(-2010)] \textit{Bob} - Bob is a signal-processing and machine
    learning toolbox.  It provides fundamental building blocks and complete
    systems for reproducible research software packages.  I'm one of Bob's main
    authors and architect.  A complete list of software packages can be browsed
    from \url{https://gitlab.idiap.ch/bob}. Notable recent packages:

    \begin{description}
      \item[2021] \textit{bob.med.tb} - Tuberculosis detection from front chest
          X-Ray images; URL: \url{https://gitlab.idiap.ch/bob/bob.med.tb}

      \item[2019] \textit{bob.ip.binseg} - Semantic Segmentation Benchmark
          Package for Bob; URL: \url{https://gitlab.idiap.ch/bob/bob.ip.binseg}

      \item[2018] \textit{bob.pad.face} - Face Presentation-Attack Detection
        algorithms; URL: \url{https://gitlab.idiap.ch/bob/bob.pad.face}

      \item[2018] \textit{bob.bio.vein} - Vein biometrics recognition
        algorithms; URL: \url{https://gitlab.idiap.ch/bob/bob.bio.vein}

      \item[2017] \textit{bob.rppg.base} - Baseline Algorithms for Remote
        Photoplethysmography (rPPG); URL:
        \url{https://gitlab.idiap.ch/bob/bob.rppg.base}

      \item[2017] \textit{bob.paper.icml2017} - Reproducible Research baseline
        algorithms for our ICML'17 paper; URL
        \url{https://gitlab.idiap.ch/bob/bob.paper.icml2017}
    \end{description}


  \item[\thecurrentyear(-2012)] \textit{BEAT} - The BEAT platform is a European
    computing e-infrastructure for Open Science.  The source code for the whole
    platform is distributed through an open-source license and maintained to
    date.  I'm one of BEAT's main authors and architect.  A complete list of
    software packages can be browsed from: \url{https://gitlab.idiap.ch/beat}.

\end{description}

\end{document}

\typeout{ *************** End of file cv *************** }
