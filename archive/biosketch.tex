%%%%%%%%%%%%%%%%%%%%%%%%%%%%%%%%%%%%%%
% coding=utf-8                       %
% Andre Anjos <andre.anjos@idiap.ch> %
% Mon Dec 13 07:42:50 CET 2021       %
%%%%%%%%%%%%%%%%%%%%%%%%%%%%%%%%%%%%%%

\documentclass[a4paper]{nihbiosketch}

% sorting=none will preserve the bib file order
\usepackage{csquotes}
\usepackage[backend=biber,maxbibnames=99,bibstyle=publist,plauthorhandling=highlight,marginyear=false,sorting=ddnt]{biblatex}
\setlength\bibitemsep{1.5\itemsep}
\plauthorname[André]{Anjos}

\addbibresource{biosketch.bib}

% This avoids the overfull boxes because biblatex does not break underlined
% titles by default.
\DeclareFieldFormat*{title}{#1}

\DeclareFieldFormat*{titlecase}{%
    \ifdef{\currentfield}
      {\ifcurrentfield{title}
         {\usefield{\uline}{\currentfield}}%
         {#1}}
      {#1}}

\usepackage{color}
\usepackage{hyperref}
\definecolor{linkcolor}{rgb}{0.1,0.1,0.4}

\AfterPreamble{\hypersetup{
  pdfauthor={Andre Anjos},
  pdftitle={Biosketch for Andre Anjos},
  pdfsubject={Biosketch for Andre Anjos},
  urlcolor=blue,
  %colorlinks=true,    % false: boxed links; true: colored links
  linkcolor=linkcolor, % color of internal links (change box color with linkbordercolor)
  citecolor=linkcolor, % color of links to bibliography
  filecolor=linkcolor, % color of file links
  urlcolor=linkcolor   % color of external links
}}

\name{Anjos, André}
%\eracommons{huntmc}
\position{Scientific Researcher}

\begin{document}

\begin{education}
    Federal University of Rio de Janeiro, Brazil &
    B.S. &
    12/1999 &
    Electrical Engineering \\

    Federal University of Rio de Janeiro, Brazil &
    M.Sc. &
    12/2001 &
    Signal Processing \\

    Federal University of Rio de Janeiro, Brazil and CERN, Switzerland &
    D.Sc. &
    12/2006 &
    Machine Learning \\

    Idiap Research Institute, Switzerland &
    Postdoctoral &
    06/2014 &
    Biometrics \\
\end{education}

\section{Personal Statement}

I received my Ph.D. degree in signal processing from the
\href{https://www.ufrj.br/}{Federal University of Rio de Janeiro} in 2006. I
joined the \href{https://atlas.ch/}{ATLAS Experiment} at European Centre for
Particle Physics (\href{https://www.cern.ch/}{CERN}, Switzerland) from 2001
until 2010 where I worked in the development and deployment of the Trigger and
Data Acquisition systems that are nowadays powering the discovery of the Higgs
boson. During my time at \href{https://www.cern.ch/}{CERN}, I studied the
application of neural networks and statistical methods for particle recognition
at the trigger level and developed several software components still in use
today. In 2010, I joined the
\href{https://www.idiap.ch/en/scientific-research/biometrics-security-and-privacy}{Biometrics
Security and Privacy Group} at the
\href{https://www.idiap.ch/en/scientific-research/biosignal-processing}{Idiap
Research Institute} where he worked with face and vein biometrics, presentation
attack detection, and reproducibility in research.  Since 2018 I head the
\href{https://www.idiap.ch/en/scientific-research/biosignal-processing}{Biosignal
Processing Group} at Idiap. My current research interests include medical
applications, image and signal processing, machine learning, research
reproducibility and open science.  Among my open-source contributions, one can
cite \href{https://www.idiap.ch/software/bob}{Bob} and the
\href{https://www.idiap.ch/software/beat}{BEAT framework} for evaluation and
testing of machine learning systems. I teach graduate-level
\href{http://edu.epfl.ch/coursebook/en/fundamentals-in-statistical-pattern-recognition-EE-612}{machine
learning courses} at the École Polytechnique Fédérale de Lausanne
(\href{https://www.epfl.ch/}{EPFL}) and master courses at Idiap's
\href{https://master-ai.ch/}{Master of AI}. I serve as reviewer for various
scientific journals in pattern recognition, machine learning, and image.

% key publications
%\begin{enumerate}
%    \item pub1
%    \item pub2
%\end{enumerate}

\section{Positions}  % and Honors

\subsection*{Positions and Employment}
\begin{datetbl}
2007--2010 & Research and Development Engineer, University of Wisconsin, Madison, USA \\

2010--2013 & Post-doctoral Researcher, Idiap Research Institute, Switzerland \\

2014--2018 & Research Associate, Idiap Research Institute, Switzerland \\

2018-- & Scientific Researcher, Idiap Research Institute, Switzerland \\
\end{datetbl}

\section{Contribution to Science}

\begin{enumerate}

\item \textbf{Computer Vision and Machine Learning} I have actively worked in
    computer vision and deep learning (mostly) associated to biometric
    recognition, with potential application to various other tasks.
    Contributions range from the collection of datasets, the exploration of
    different methods to address and assess biometric recognition
    vulnerabilities, domain adaptation, and remote photoplethysmography.

    While I have worked in various contributions, I highlight here key
    publications from the past 5 years that may related to the proposal.

    \begin{itemize}

        \item Domain Specific Units (Adaptation): In~\cite{tifs-2019}, we apply
            domain adaptation via dedicated Domain-Specific Units (DSU), with
            an application to Heterogeneous Face Recognition.  In this class of
            problems, one wants to recognize an individual across different
            spectral data,  based on the representation on a principal spectrum
            (e.g., visual).  It is a challenging task because multi-spectral
            data for covering large populations is rare, which in turn stymies
            the training of deep convolution-based architectures for this task.
            We developed a mechanism to adapt the parameters of models
            pre-trained on large visual spectral face recognition datasets,
            which are readily available.  My contributions are directly related
            to core idea of this work.

      \item Anomaly Detection and Robustness: Biometric recognition systems are
          exposed to presentation attacks, and dectectors (PAD) for this
          purpose are required building blocks of thrustworthy systems.  Most
          PAD systems work discriminatively, trying to separate attacks from
          \textit{bona fide} presentations.  We showed this technique does not
          generalize well to unseen presentation attacks.  We explored, for the
          first time, alternate approaches by joint-modelling client identity
          as a way to calibrate PAD output scores~\cite{tifs-2015}, showing
          increased robustness to unseen attacks.  More recently, we also
          showed that solely modelling \textit{bona fide} presentations is also
          an effective way to increased PAD robustness~\cite{icb-2018}.
          Finally, we showed that by adding heterogenous inputs to PAD systems
          can improve their robustness~\cite{tifs-2019-2} to achieve
          state-of-the-art performance, even to unseen conditions during
          training.

    \end{itemize}

    This work was published as book chapters, international peer-reviewed
    scientific journals (including articles at very high-impact factor
    journals) and in peer-reviewed conference papers totalling a few thousand
    citations. Accompanying software packages for suh contributions were
    released publicly, under an open-source license. For details and more
    links, please refer to the applicant's Research Output.

\item \textbf{Semantic Segmentation for Medical Imaging}: Since the
    introduction of U-Nets in 2015, the field of medical image segmentation has
    seen renewed interest bringing in a variety of fully convolutional (deep)
    neural network (FCN) architectures for binary and multi-class segmentation
    problems promising very attractive results, with applications in computed
    tomography, retinography, and histopathology to cite a few.  Despite the
    incredible progress, the lack of annotated images (due to cost), and rigor
    in the comparison of trained models has led the community to believe larger
    and more dense networks provide better results.  This is particularly
    noticeable in ophtalmological images such as those from bi-dimensional eye
    fundus photography (retinography).  While retinography is not used for
    precision diagnosis, it remains cheap and very effective means for mass
    screening.  Semantical segmentation of eye fundus structures plays a key
    role in this process.

    I tried to address these gaps in two different ways.  The
    first~\cite{arxiv-2019} was to conduct and publish rigorous (open source,
    reproducible) benchmarks with popular retinography datasets and
    state-of-the-art FCN models in which we: i) showed that simple
    transformation techniques like rescaling, padding and cropping of combined
    lower-resolution source datasets to the resolution and spatial composition
    of a higher-resolution target dataset can be a surprisingly effective way
    to improve segmentation quality in unseen conditions; ii) we proposed a set
    of plots and metrics that give additional insights into model performance
    and demonstrated via tables and plots how to take advantage of that
    information, throwing a new light over some published benchmarks. In a
    second contribution~\cite{arxiv-2020} we propose that a minimalistic
    version of a standard U-Net with 3 orders of magnitude less parameters,
    carefully trained and rigorously evaluated, closely approximates the
    state-of-the-art performance in vessel segmentation for retinography.  In
    addition, we propose a simple extension, dubbed W-Net, by concatenating two
    U-Nets together, which reaches outstanding performance on several popular
    datasets, still using orders of magnitude less learnable weights than any
    previously published approach.

\item \textbf{Reproducible Research}: We've been actively looking at the
    reproducibility of published work and how to lower the entrance barrier of
    publication readers, converting them into engaged users of methods we
    create.  We argue it is insufficient, in most cases, to only publish
    software leading to results if original data remains inaccessible. In
    particular~\cite{acmmm-2012}, we note that reproducibility should imply in
    the following characteristics: repeatability, shareability, extensibility
    and stability, which is not guaranteed by most published material to date.
    We proposed an open-source software suite called Bob
    (\url{https://www.idiap.ch/software/bob}) that possesses such
    characteristics, demonstrating its flexibility to various tasks including
    Medical Image Segmentation
    (\url{https://gitlab.idiap.ch/bob/bob.ip.binseg}), Remote
    Photoplethysmography (\url{https://gitlab.idiap.ch/bob/bob.rppg.base}), and
    Biometric Person Recognition
    (\url{https://gitlab.idiap.ch/bob/bob.bio.base}).

    From another perspective, there are legitimate cases in which raw data
    leading to research conclusions cannot be published.  Furthermore, in a
    growing number of use-cases, the availability of both software does not
    translate to an accessible reproducibility scenario.  The user, for
    example, may not have the necessary equipment to perform the analysis.  To
    bridge this gap, we built an open platform for research
    (\url{https://www.beat-eu.org/platform}) in computational sciences related
    to pattern recognition and machine learning, to help on the development,
    reproducibility and certification of results obtained in the
    field~\cite{icml-2017-1}.  The BEAT platform is distributed under an
    open-source license (\url{https://www.idiap.ch/software/beat/}).

    Both projects are still active and support past and future work at Idiap
    and beyond.  We conducted (and will continue doing) lectures to both master
    and graduate students about reproducibility in data science (refer to the
    applicant's CV under the section "Teaching" for details).  There are
    currently $\sim$180 \textit{direct} citations to publications about Bob and
    BEAT core frameworks.

\end{enumerate}

\renewcommand{\refname}{Referenced Publications}
\nocite{*}
\printbibliography

\subsection*{Complete List of Published Work:}
\url{https://anjos.ai/publications/}

\section{Research Support}

\subsection*{Ongoing Research Support}


\grantinfo{CH SNSF}{FairMI}{11/2023--10/2027}
{Machine Learning Fairness for Medical Images}
{The goal of this project is to create machine learning models that are more
equitable to different demographies.}
{Role: PI}

\bigskip

\grantinfo{CH SNSF}{CollabCloud}{01/2024--12/2024}
{A Collaborative Research Cloud Infrastructure powering Discovery and Exchange}
{The goal of this project is to develop a collaborative and research platform
replacing the current infrastructure available at the Idiap Research Institute.}
{Role: PI}

\bigskip

\grantinfo{CH TheArk}{SECure}{01/2022--02/2024}
{Safe and Explainable Clinical AI for Orthopaedic Surgical Assessment}
{The goal of this project is to build a predictive tool to support medical
decision in surgery assessment.}
{Role: PI}

\subsection*{Completed Research Support}

\grantinfo{EU H2020}{AI4EU}{01/2019--02/2022}
{Artitificial Intelligence Platform for Europe}
{The goal of this project is the design and construction of an Artificial
    Intelligence platform assembling all stakeholders in Europe.}
{Role: Partner}

\bigskip

\grantinfo{EU CHIST-ERA}{LEARN-REAL}{01/2019--12/2022}
{Learning physical manipulation skills with simulators using realistic variations}
{The goal of this project is to build reproducible benchmarks of realistic
robotic simulations that can be used by the community.}
{Role: co-PI}

\bigskip

\grantinfo{EU CHIST-ERA}{ALLIES}{01/2018--12/2021}
{Autonomous Lifelong learnIng intelLigent Systems}
{The goal of this project is to build machine learning systems that can learn
through time.}
{Role: Partner}

\end{document}
