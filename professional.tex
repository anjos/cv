\typeout{ ====================================================================}
\typeout{ This is file professional, created at Tue 22 Jan 2013 10:32:37 CET }
\typeout{ Andre Anjos <andre.anjos@idiap.ch> }
\typeout{ ====================================================================}

\section{Professional Experience}

%\cventry{year--year}{Job title}{Employer}{City}{}{General description no longer than 1--2 lines.\newline{}
%Detailed achievements:
%\begin{itemize}%
%\item Achievement 1;
%\item Achievement 2, with sub-achievements:
%  \begin{itemize}
%  \item Sub-achievement (a);
%  \item Sub-achievement (b), with sub-sub-achievements (don't do this!);
%    \begin{itemize}
%    \item Sub-sub-achievement i;
%    \item Sub-sub-achievement ii;
%    \item Sub-sub-achievement iii;
%    \end{itemize}
%  \item Sub-achievement (c);
%  \end{itemize}
%\item Achievement 3.
%\end{itemize}}

\cventry{2018}{Permanent Researcher}{}{Idiap Research
Institute, Martigny, Switzerland}{}{Head of the Biosignal Processing Group.\newline{}
\textbf{Teaching}:
\begin{itemize}
  \item \href{https://master-ai.ch}{Master in Artificial intelligence}, post-graduate level, Idiap/Wallis/Unidistance partnership, Academic Supervisor for Modules M05 (Open Science and Ethics), M06/M08 (Fundamentals of Machine Learning 1 and 2)
  \item \href{http://edu.epfl.ch/coursebook/en/fundamentals-in-statistical-pattern-recognition-EE-612}{Fundamentals
    in statistical pattern recognition}, post-graduate level, at the École
    Polytechnique Fédérale de Lausanne (EPLF, Switzerland), EE-612
\end{itemize}
\textbf{Supervision}:
\begin{itemize}
\item Supervision of Colombine Verzat, master student on Adverse event
    Detection for Latent Tuberculosis Infection (LTBI) Treatment
\item Supervision of Tim Laibacher, graduate student on 2D Eye-fundus Binary Segmentation
\item Supervision of Jaden Diefenbaugh, graduate student on work associated with the \href{https://www.beat-eu.org/platform}{BEAT platform}
\end{itemize}
\textbf{Interests}:
\begin{itemize}
  \item Biomedical-related signal processing applications
  \item Analysis of biomedical records and e-Health
  \item Diagnostics for medical-related applications
  \item Image and signal processing
  \item Pattern recognition and machine learning
  \item Open-science
\end{itemize}}

\cventry{2014--2018}{Research Associate}{}{Idiap Research
Institute, Martigny, Switzerland}{}{Research on Biometrics, Security and
Computing.\newline{}
\textbf{Teaching}:
\begin{itemize}
  \item \href{http://edu.epfl.ch/coursebook/en/fundamentals-in-statistical-pattern-recognition-EE-612}{Fundamentals
    in statistical pattern recognition}, post-graduate level, at the École
    Polytechnique Fédérale de Lausanne (EPLF, Switzerland), EE-612
  \item Guest Lecturer at State University of São Paulo, Campus Bauru, for the
    master level course ``Special Topics in Reproducible Research Pattern
    Recognition and Machine Learning", June 2015
\end{itemize}
\textbf{Supervision}:
\begin{itemize}
\item Co-supervision of \href{http://www.idiap.ch/~ichingo/}{Ivana Chingovska},
doctorate student.
\end{itemize}
\textbf{Project writing and involvement}:
\begin{itemize}
\item \href{http://www.beat-eu.org/platform/}{FP7 BEAT}: contact point for
  Idiap, leader on 2 work packages. BEAT develops a new
  \href{https://gitlab.idiap.ch/beat/}{open-source} online platform for
  biometric system certification and development;
\item CTI Project FEDARS: face recognition using deep architectures;
\item Hassler Project COHFACE: remote photo-plethysmography (heart-rate
  measurements) from webcam images, application to face anti-spoofing;
\item CTI Project 3DFingerVein: vein recognition using a finger vein imagery
  acquired from multiple cameras; Technical lead at Idiap;
\item IARPA Project Odin (BATL team): presentation attack detection for face
  recognition using Visual, NIR and Thermal cameras. Technical contact point;
\item Development and management of
\href{http://www.idiap.ch/software/bob/}{Bob}, a framework for reproducible
research in pattern recognition and machine learning.
\end{itemize}}

\cventry{2010--2013}{Post-doctoral Researcher}{}{Idiap Research
Institute, Martigny, Switzerland}{}{Research on Biometrics, Security and
Computing.\newline{}
\textbf{Teaching}:
\begin{itemize}
  \item \href{http://edu.epfl.ch/coursebook/en/fundamentals-in-statistical-pattern-recognition-EE-612}{Fundamentals in statistical pattern recognition}, post-graduate level, at the École Polytechnique Fédérale de Lausanne (EPLF, Switzerland), EE-612
\end{itemize}
\textbf{Supervision}:
\begin{itemize}
\item Co-supervision of \href{http://www.idiap.ch/~ichingo/}{Ivana Chingovska},
doctorate student;
\item Co-supervision of Tiago Freitas Pereira and Jükka Mäatta, visiting
students;
\item Co-supervision of Murali Mohan Chakka, trainee.
\end{itemize}
\textbf{Project writing and involvement}:
\begin{itemize}
\item \href{http://www.beat-eu.org/}{FP7 BEAT}: contact point for Idiap, leader
on 2 work packages. BEAT develops a new online platform for biometric system
certification and development;
\item \href{http://www.tabularasa-euproject.org/}{FP7 TABULA RASA}: contact
point for Idiap. TABULA RASA investigates sensitiveness to spoofing attacks on
biometric systems;
\item CTI Project Replay: development of database and counter-measures to
spoofing. Construction of working prototypes with the industrial partner
(KeyLemon);
\item Development and management of
\href{http://www.idiap.ch/software/bob/}{Bob}, a framework for reproducible
research.
\end{itemize}}

\cventry{2004--2010}{Researcher}{}{University of Wisconsin, Madison, USA}{}{Development and construction of the ATLAS
Trigger and Data-Acquisition Systems, at \href{http://www.cern.ch}{CERN},
Switzerland.\newline{}
\textbf{Supervision}:
\begin{itemize}
\item Supervision of CERN summer students (three summers);
\item Co-supervision Rodrigo Coura Torres, doctorate student.
\end{itemize}
\textbf{Achievements}:
\begin{itemize}
\item Design, development and maintenance of the ATLAS Trigger and Data
Acquisition systems. It is currently deployed and operational on more than 5000
machines at \href{http://www.cern.ch}{CERN} and worldwide.
\end{itemize}}

\cventry{2000--2004}{Graduate Student}{}{Federal University of Rio de
Janeiro, Brazil}{}{Design and development of a novel algorithm for
fast physics triggering based on Calorimetry, Topological Mapping and Neural
Networks for the \href{http://atlas.ch}{ATLAS Experiment},
\href{http://www.cern.ch}{CERN, Switzerland}.}

\cventry{1994--1999}{Under Graduate Student}{}{Federal University of Rio de
Janeiro, Brazil (includes 1 year intership at \href{http://www.cern.ch}{CERN,
Switzerland})}{}{Signal Processing student.\newline{}
\textbf{Achievements}:
\begin{itemize}
\item Sub-optimal filtering for particle discrimination;
\item Neural classifiers implemented in a transputer-based parallel machine;
\item Spent 1 year at \href{http://www.cern.ch}{CERN, Switzerland} in a joint
project with the university. Participated in the development of early
prototypes of the 2nd level-trigger and what would become the final design
of the trigger for the ATLAS experiment.
\end{itemize}
}

\typeout{ *************** End of file professional *************** }
