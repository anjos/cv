\typeout{ ====================================================================}
\typeout{ This is file snsf-2p-cv, created at Mon 14 May 11:45:34 2018 CEST   }
\typeout{ Andre Anjos <andre.anjos@idiap.ch>                                  }
\typeout{ ====================================================================}

\documentclass[11pt,a4paper,sans]{moderncv}

% styles: 'casual' (default), 'classic', 'oldstyle','banking'
\moderncvstyle{casual}

% colors: 'blue' (default), 'orange', 'green', 'red', 'purple', 'grey', 'black'
\moderncvcolor{blue}

\usepackage[utf8]{inputenc}
\usepackage{textcomp}

% adjust the page margins
\usepackage[scale=0.8]{geometry}
% if you want to change the width of the column with the dates
%\setlength{\hintscolumnwidth}{3cm}

% for the 'classic' style, if you want to force the width allocated to your
% name and avoid line breaks. be careful though, the length is normally
% calculated to avoid any overlap with your personal info; use this at your own
% typographical risks...

%\setlength{\makecvtitlenamewidth}{10cm}

% hyperref setup
\definecolor{linkcolor}{rgb}{0.1,0.1,0.4}

\AfterPreamble{\hypersetup{
  pdfauthor={Andre Anjos},
  pdftitle={Short Curriculum Vitate for Andre Anjos},
  pdfsubject={Short Curriculum Vitae for Andre Anjos},
  urlcolor=blue,
  colorlinks=true,    % false: boxed links; true: colored links
  linkcolor=linkcolor, % color of internal links (change box color with linkbordercolor)
  citecolor=linkcolor, % color of links to bibliography
  filecolor=linkcolor, % color of file links
  urlcolor=linkcolor   % color of external links
}}

% personal data
\firstname{André}
\familyname{Anjos}

% optional, remove / comment the line if not wanted
\address{Idiap Research Institute, rue Marconi 19, Centre du Parc, Martigny,
Switzerland, CH-1920}{}

% optional, remove / comment the line if not wanted
\phone{+41277217763}

% optional, remove / comment the line if not wanted
\email{andre.anjos@idiap.ch}

% optional, remove / comment the line if not wanted
\homepage{andreanjos.org}

% optional, remove / comment the line if not wanted
%\extrainfo{}

% optional, remove / comment the line if not wanted
%\quote{Some quote}

\begin{document}

\section{André Anjos -- Permanent Researcher in Biosignal Processing}

\cvlanguage{Date of Birth:}{December 31, 1973}{}
\cvlanguage{Nationality:}{Swiss and Brazilian}{}
\cvlanguage{Address:}{Idiap Research Institute, rue Marconi 19, CH-1920 Martigny, Switzerland}{}
\cvlanguage{Phone:}{+41277217763}{}
\cvlanguage{E-mail:}{andre.anjos@idiap.ch}{}
\cvlanguage{Website:}{\url{http://andreanjos.org}}{}
\cvlanguage{H-index:}{21}{}
\cvlanguage{Scholar:}{\url{https://scholar.google.ch/citations?user=pAfLhMoAAAAJ&hl=en}}{}

\section{Education}

\cventry{2001--2006}{Doctor}{}{\href{http://www.ufrj.br}{Federal
University of Rio de Janeiro}, Brazil}{}{D.Sc. degree from the
\href{http://www.pee.ufrj.br/}{Electronics Engineering School}
(\href{https://www.lps.ufrj.br/}{Signal Processing Laboratory}),
\href{http://www.coppe.ufrj.br/}{Post Graduate Program (COPPE)}.
Thesis work developed at \href{http://www.cern.ch}{CERN, Switzerland} in the
context of the \href{http://www.cern.ch/atlas/}{ATLAS Experiment}.}

\cventry{2000--2001}{Master}{}{\href{http://www.ufrj.br}{Federal
University of Rio de Janeiro}, Brazil}{}{M.Sc. degree from the
\href{http://www.pee.ufrj.br/}{Electronics Engineering School}
(\href{https://www.lps.ufrj.br/}{Signal Processing Laboratory}),
\href{http://www.coppe.ufrj.br/}{Post Graduate Program of the Federal
University of Rio de Janeiro (COPPE}/UFRJ). Thesis work developed at the
university in the context of the
\href{http://www.cern.ch/atlas/}{ATLAS-CERN}/UFRJ collaboration.}

\cventry{1994--1999}{Engineer}{}{\href{http://www.ufrj.br}{Federal
University of Rio de Janeiro}, Brazil}{}{Electronics Engineering degree from
the \href{http://www.del.ufrj.br}{Electronics Engineering Department}.}

\section{Employment History}

\cventry{2018}{Permanent Researcher}{}{Idiap Research Institute, Martigny, Switzerland}{}{Head of the Biosignal Processing Group.}

\cventry{2014--2018}{Research Associate}{}{Idiap Research Institute, Martigny, Switzerland}{}{Research on Biometrics, Security and Computing.}

\cventry{2010--2013}{Post-doctoral Researcher}{}{Idiap Research Institute, Martigny, Switzerland}{}{Research on Biometrics, Security and Computing.}

\cventry{2004--2010}{Researcher}{}{University of Wisconsin, Madison, USA}{}{Development and construction of the ATLAS Trigger and Data-Acquisition Systems, at \href{http://www.cern.ch}{CERN}, Switzerland.}

\section{Teaching}

\cventry{}{\href{http://edu.epfl.ch/coursebook/en/fundamentals-in-statistical-pattern-recognition-EE-612}{Fundamentals in statistical pattern recognition}}{post-graduate level, at the École Polytechnique Fédérale de Lausanne (EPFL, Switzerland), EE-612}{EDEE Program in Spring 2013 (11 students), Spring 2015 (20 students) and Sprint 2017 (29 students)}{}{}
\cventry{2016}{Reproducible Research through Bob and the BEAT
Platform}{Special Topic for International Master on Biometrics}{Université Paris-Est Créteil (UPEC)}{}{}
\cventry{2015}{Special Topics in Reproducible Research Pattern Recognition and Machine Learning}{Guest Lecturer at State University of São Paulo, master level course}{}{}{}

\section{Student Supervision}

\cventry{2016--2017}{Graduate Student}{Jaden Diefenbaugh}{Reproducible Research}{Work associated with the \href{https://www.beat-eu.org/platform}{BEAT platform}}{}
\cventry{2011--2015}{Doctoral Student}{\href{http://www.idiap.ch/~ichingo/}{Ivana Chingovska}}{Thesis work associated with the FP7 TABULA RASA project}{}{}
\cventry{2012}{Graduate Student}{Tiago de Freitas Pereira}{Biometric presentation attack detection}{Visiting student}{Part of master thesis}{}
\cventry{2012}{Graduate Student}{Jukka Mäatta}{Biometric presentation attack detection}{Visiting student}{Part of Ph.D thesis}
\cventry{2011}{Graduate Student}{Murali Mohan Chakka}{}{}{}

\section{Scientific Contributions}

\subsection{Open Software}

\cventry{2010--}{Bob}{\url{https://www.idiap.ch/software/bob}}{A framework for
reproducible research in pattern recognition and machine learning}{}{}

\cventry{2012--}{BEAT}{\url{https://www.beat-eu.org/platform}}{A platform for open-science in machine learning and pattern recogntion}{}{}

\subsection{Open Data}

\cventry{2016}{COHFACE}{\url{https://www.idiap.ch/dataset/cohface}}{An open-dataset for studying remote photo-plethysmography}{}{}

\cventry{2011}{Replay-Attack}{\url{https://www.idiap.ch/dataset/replayattack}}{A video dataset for studing presentation-attack detection in Biometrics}{}{}

\subsection{Committees}

\cventry{2016}{IAPR International Conference on Biometrics}{Program Committee}{}{}{}
\cventry{2015}{IEEE International Conference on Biometrics: Theory, Applications, and Systems}{Participation on Doctoral Consortium}{}{}{}
\cventry{2015}{IEEE International Conference on Biometrics: Theory, Applications, and Systems}{Chair of special session on ``Reproducible Research in Biometrics"}{}{}{}
\cventry{2015}{IAPR International Conference on Biometrics}{Program Committee}{}{}{}
\cventry{2014}{International Workshop on Soft-biometrics}{Program Committee}{Hosted by ECCV}{}{}

\subsection{Reviewer}

\cventry{2014--}{Elsevier Image and Vision Computing (IMAVIS)}{}{}{}{}
\cventry{2013--}{IEEE Transactions on Information Forensics and Security}{}{}{}{}
\cventry{2013--}{IET Biometrics}{}{}{}{}
\cventry{2012--}{IEEE Transactions in Image Processing}{}{}{}{}
\cventry{2012--}{IEEE Transactions on Circuits and Systems for Video Technology}{}{}{}{}

\end{document}

\typeout{ *************** End of file cv *************** }
