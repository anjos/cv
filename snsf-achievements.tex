\typeout{ ====================================================================}
\typeout{ This is file snsf-achievements, created at Mon 14 May 11:45:34 2018 CEST }
\typeout{ Andre Anjos <andre.anjos@idiap.ch>                                  }
\typeout{ ====================================================================}

\documentclass[a4paper,10pt,onecolumn]{article}
\usepackage[T1]{fontenc}

% Choose your font here - combinations with harmonized math symbols suggested
%\usepackage{lmodern} %Times-like
\usepackage{newpxtext} \usepackage[euler-digits]{eulervm} %Palatino-like
%\usepackage{libertine} \usepackage[libertine]{newtxmath} %Libertine-like

% adjust the page margins
\usepackage[scale=0.8]{geometry}

\usepackage{hyperref}
\usepackage{titling}

\pretitle{\vspace{-6em}\begin{flushleft}\LARGE}
\posttitle{\par\end{flushleft}}
\preauthor{}
\postauthor{}
\author{}
\predate{}
\postdate{}
\date{}

\hypersetup{
backref,%
colorlinks,%
citecolor=black,%
filecolor=black,%
linkcolor=black,%
urlcolor=black
}


\begin{document}

% In the section Major scientific achievements (max. 2 pages to be attached to
% the CV) the applicant describes his/her most important scientific
% achievements in the past five years.  Older achievements can be listed, if
% they are of particularly high importance for the proposal. In addition to
% scientific publications, any other relevant information, such as a knowledge
% transfer event, a software, database, prototype, etc. may be provided here.
% Describe for each achievement the applicant’s specific contribution and the
% overall impact of the work.

% All these items are intended for evaluators to specifically assess the
% scientific quality and relevance of the research output. If possible, the
% documents/sources of the various achievements should be made available to the
% evaluators via a direct open access web-link.

\title{Major Scientific Achievements (Dr. André Anjos)}
\maketitle

\section{Biometric Anti-Spoofing}

We advanced the state-of-the-art in presentation attack detection (PAD) through
various contributions, ranging from the exploration of different methods to
address PAD and assess biometric recognition vulnerabilities, to publishing
articles and book chapters, datasets and software for easy reproducibility of
claimed results.  My contributions to biometric anti-spoofing in the past five
years can be summarized as follows:

\begin{itemize}

  \item Anomaly Detection and Robustness: Most PAD systems work
    discriminatively, trying to separate attacks from \textit{bona fide}
    presentations.  We showed this technique does not generalize well to unseen
    presentation attacks.  We explored, for the first time, alternate
    approaches by joint-modelling client identity as a way to calibrate PAD
    output scores\footnote{See \textit{``On the use of client identity
    information for face anti-spoofing"}, Feb 2015", Access:,
    \url{http://publications.idiap.ch/index.php/publications/show/3069}},
    showing increased robustness to unseen attacks.  More recently, we also
    showed that solely modelling \textit{bona fide} presentations is also an
    effective way to increased PAD robustness\footnote{See \textit{``On
    Effectiveness of Anomaly Detection Approaches against Unseen Presentation
    Attacks in Face Anti-Spoofing"}, Feb. 2018, Access:
    \url{https://publidiap.idiap.ch/index.php/publications/show/3793}}.

  \item Vulnerability Analysis and Evaluation: We demonstrated candidate
    presentation attack instruments and biometric recognition systems must
    undergo a vulnerability assessment to quantify the pertinence and quality
    of attacks\footnote{See \textit{``Biometrics Evaluation under Spoofing
    Attacks"}, 2014, DOI:
    \url{https://publidiap.idiap.ch/index.php/publications/show/2922}}.  We
    also devised a way to tune protected biometric systems to various different
    applications, with changing security requirements.

  \item Visual and Multi-spectral PAD: Finally, we showed it is possible to
    add sensors to PAD systems to improve their robustness\footnote{See
    \textit{``Face Recognition Systems Under Spoofing Attacks"}, Feb. 2016, DOI:
    \url{http://dx.doi.org/10.1007/978-3-319-28501-6_8}}.

\end{itemize}

This work was published as book chapters, international peer-reviewed
scientific journals (including 2 articles at IEEE Transactions on Information
Forensics and Security with an impact factor of 4.3) and in peer-reviewed
conference papers totalling about 200 citations.  Accompanying software
packages for such contributions were released publicly, under an open-source
license. For details and more links, please refer to the applicant's Research
Output.

\section{Reproducible Research}

We've been actively looking at the reproducibility of published work and how to
lower the entrance barrier of publication readers, converting them into engaged
users of methods we create.  We argue it is insufficient, in most cases, to
only publish software leading to results if original data remains inaccessible.
In particular\footnote{See \textit{``Continuously Reproducing Toolchains in
Pattern Recognition and Machine Learning Experiments"}, Aug. 2017, Access:
\url{https://publications.idiap.ch/index.php/publications/show/3666}}, we note
that reproducibility should imply in the following characteristics:
repeatability, shareability, extensibility and stability, which is not
guaranteed by most published material to date.  We proposed a software suite
called Bob (\url{https://www.idiap.ch/software/bob}) that possesses such such
characteristics, demonstrating its flexibility to various tasks including
Biometric Person Recognition (\url{https://gitlab.idiap.ch/bob/bob.bio.base}),
Presentation Attack Detection (\url{https://gitlab.idiap.ch/bob/bob.pad.base}),
Remote Photoplethysmography (\url{https://gitlab.idiap.ch/bob/bob.rppg.base}),
Speech Processing (\url{https://gitlab.idiap.ch/bob/bob.kaldi}).  All
contributions are distributed under an open-source license.

From another perspective, there are legitimate cases in which raw data leading
to research conclusions cannot be published.  Furthermore, in a growing number
of use-cases, the availability of both software does not translate to an
accessible reproducibility scenario.  The user, for example, may not have the
necessary equipment to perform the analysis.  To bridge this gap, we built an
open platform for research\footnote{\url{https:www.beat-eu.org/platform}} in
computational sciences related to pattern recognition and machine learning, to
help on the development, reproducibility and certification of results obtained
in the field\footnote{See \textit{``BEAT: An Open-Science Web Platform"}, Aug.
2017, Access:
\url{https://publidiap.idiap.ch/index.php/publications/show/3665}}.  The BEAT
platform is distributed under an open-source
license\footnote{\url{https://gitlab.idiap.ch/beat/}}.

Both projects are still active and support past and future work at Idiap and
beyond.  We conducted (and will continue doing) lectures to both master and
graduate students about reproducibility in data science (refer to the
applicant's CV under the section ``Teaching" for details).  There are currently
$\sim$120 \textit{direct} citations to publications about Bob and BEAT core
frameworks.

\end{document}

\typeout{ *************** End of file cv *************** }
