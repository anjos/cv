\typeout{ ====================================================================}
\typeout{ This is file euraxess-cv, created at Mon 14 May 11:45:34 2018 CEST  }
\typeout{ Andre Anjos <andre.anjos@idiap.ch>                                  }
\typeout{ ====================================================================}

\documentclass[11pt,a4paper,sans]{moderncv}
\usepackage[T1]{fontenc}

% styles: 'casual' (default), 'classic', 'oldstyle','banking'
\moderncvstyle{casual}

% colors: 'blue' (default), 'orange', 'green', 'red', 'purple', 'grey', 'black'
\moderncvcolor{blue}

% sorting=none will preserve the bib file order
\usepackage{csquotes}
\usepackage[backend=biber,maxbibnames=99,bibstyle=publist,plauthorhandling=highlight,marginyear=false,plnumbered=false,sorting=ddnt]{biblatex}
\setlength\bibitemsep{1.5\itemsep}
\plauthorname[André]{Anjos}

\addbibresource{fapesp.bib}

% adjust the page margins
\usepackage[margin=2cm,left=2.5cm]{geometry}
% if you want to change the width of the column with the dates
%\setlength{\hintscolumnwidth}{3cm}

% for the 'classic' style, if you want to force the width allocated to your
% name and avoid line breaks. be careful though, the length is normally
% calculated to avoid any overlap with your personal info; use this at your own
% typographical risks...

%\setlength{\makecvtitlenamewidth}{10cm}

% hyperref setup
\definecolor{linkcolor}{rgb}{0.1,0.1,0.4}

\AfterPreamble{\hypersetup{
  pdfauthor={Andre Anjos},
  pdftitle={Short Curriculum Vitae for Andre Anjos},
  pdfsubject={Short Curriculum Vitae for Andre Anjos},
  urlcolor=blue,
  colorlinks=true,    % false: boxed links; true: colored links
  linkcolor=linkcolor, % color of internal links (change box color with linkbordercolor)
  citecolor=linkcolor, % color of links to bibliography
  filecolor=linkcolor, % color of file links
  urlcolor=linkcolor   % color of external links
}}

% personal data
\firstname{André}
\familyname{Anjos}

% optional, remove / comment the line if not wanted
\address{Idiap Research Institute, rue Marconi 19, Centre du Parc, Martigny,
Switzerland, CH-1920}{}

% optional, remove / comment the line if not wanted
\phone{+41277217763}

% optional, remove / comment the line if not wanted
\email{andre.anjos@idiap.ch}

% optional, remove / comment the line if not wanted
\homepage{andreanjos.org}

% optional, remove / comment the line if not wanted
%\extrainfo{}

% optional, remove / comment the line if not wanted
%\quote{Some quote}

\begin{document}

\section{André Anjos -- Researcher, Head of Biosignal Processing Group}

André Anjos received his Ph.D. degree in signal processing from the Federal
University of Rio de Janeiro in 2006. He joined the ATLAS Experiment at
European Centre for Particle Physics (CERN, Switzerland) from 2001 until 2010
where he worked in the development and deployment of the Trigger and Data
Acquisition systems that are nowadays powering the discovery of the Higgs
boson. During his time at CERN, André studied the application of neural
networks and statistical methods for particle recognition at the trigger level
and developed several software components still in use today. In 2010, André
joined the Biometrics Security and Privacy Group at the Idiap Research
Institute where he worked with face and vein biometrics, presentation attack
detection, and reproducibility in research. Since 2018 André heads the
Biosignal Processing Group at Idiap. His current research interests include
medical applications, biometrics, image and signal processing, machine
learning, research reproducibility and open science. Among André's open-source
contributions, one can cite Bob and the the BEAT framework for evaluation and
testing of machine learning systems. He teaches graduate-level machine learning
courses at the École Polytechnique Fédérale de Lausanne (EPFL) and master
courses at Idiap's Master of AI. He serves as reviewer for various scientific
journals in pattern recognition, machine learning, and image.

\typeout{ ====================================================================}
\typeout{ This is file education, created at Tue 22 Jan 2013 09:37:23 CET }
\typeout{ Andre Anjos <andre.anjos@idiap.ch> }
\typeout{ ====================================================================}

\section{Education}

% arguments 3 to 6 can be left empty
%\cventry{year--year}{Degree}{Institution}{City}{Grade}{Description}

\cventry{2001--2006}{Doctor}{}{\href{http://www.ufrj.br}{Federal
University of Rio de Janeiro}, Brazil}{}{D.Sc. degree from the
\href{http://www.pee.ufrj.br/}{Electronics Engineering School}
(\href{https://www.lps.ufrj.br/}{Signal Processing Laboratory}),
\href{http://www.coppe.ufrj.br/}{Post Graduate Program (COPPE)}.
Thesis work developed at \href{http://www.cern.ch}{CERN, Switzerland} in the
context of the \href{http://www.cern.ch/atlas/}{ATLAS Experiment}.}

\cventry{2000--2001}{Master}{}{\href{http://www.ufrj.br}{Federal
University of Rio de Janeiro}, Brazil}{}{M.Sc. degree from the
\href{http://www.pee.ufrj.br/}{Electronics Engineering School}
(\href{https://www.lps.ufrj.br/}{Signal Processing Laboratory}),
\href{http://www.coppe.ufrj.br/}{Post Graduate Program of the Federal
University of Rio de Janeiro (COPPE}/UFRJ). Thesis work developed at the
university in the context of the
\href{http://www.cern.ch/atlas/}{ATLAS-CERN}/UFRJ collaboration.}

\cventry{1994--1999}{Engineer}{}{\href{http://www.ufrj.br}{Federal
University of Rio de Janeiro}, Brazil}{}{Electronics Engineering degree from
the \href{http://www.del.ufrj.br}{Electronics Engineering Department}.}

\cventry{1989--1993}{Technician}{}{\href{http://portal.cefet-rj.br/}%
{Federal Center of Technology, Rio de
Janeiro}, Brazil}{}{Technical course in Electronics, taken at the same time as
my high-school studies.}

\typeout{ *************** End of file education *************** }



\section{Professional experience}

\cventry{2018--}{Permanent Researcher}{}{Idiap Research Institute, Martigny, Switzerland}{}{Head of the Biosignal Processing Group.}

\cventry{2014--2018}{Research Associate}{}{Idiap Research Institute, Martigny, Switzerland}{}{Research on Biometrics, Security and Computing.}

\cventry{2010--2013}{Post-doctoral Researcher}{}{Idiap Research Institute, Martigny, Switzerland}{}{Research on Biometrics, Security and Computing.}

\cventry{2004--2010}{Researcher}{}{University of Wisconsin, Madison, USA}{}{Development and construction of the ATLAS Trigger and Data-Acquisition Systems, at \href{http://www.cern.ch}{CERN}, Switzerland.}

\renewcommand{\refname}{Relevant Publications}
\nocite{*}
\printbibliography

\section{Current Research Grants}


\cventry{2019--2022}{EU H2020 "AI4EU"}{PI: Patrick Gatelier (Thales SA, FR)}{My role: Partner}{}{Budget: 20'000'000 EUR (419'621 CHF for Idiap)}

\cventry{2019--2022}{EU CHIST-ERA "LEARN-REAL"}{PI: Sylvain Calinon (Idiap, CH)}{My role: Co-PI}{}{Budget: 775'837 CHF (250'000 CHF for Idiap)}

\cventry{2018--2020}{EU CHIST-ERA "ALLIES"}{PI: Anthony Larcher (UNIMANS, FR)}{My role: Partner}{}{Budget: 496'620CHF (496'620 CHF for Idiap)}

\section{Current Student Supervisions}

\cventry{2020--2021}{Geoffrey Raposo}{Master thesis: \textit{Active tuberculosis exclusion from frontal chest X-ray images}}{}{}{}

\cventry{2019--2020}{Colombine Verzat}{Master thesis: \textit{Machine Learning for Adverse Event Detection in Latent Tuberculosis Infection Treatment}}{}{}{}

\section{Academic Indicators}

\cventry{1}{Peer-Reviewed Journals}{28}{}{}{}
\cventry{2}{Peer-Reviewed Conferences}{58}{}{}{}
\cventry{3}{Book Chapters}{9}{}{}{}
\cventry{4}{Master Thesis}{3}{}{}{}
\cventry{5}{Ph.D Thesis}{1}{}{}{}
\cventry{6}{Citations}{+19000 (h-index: 25)}{}{}{}
\cventry{7}{Patents}{3}{}{}{}

\section{Links}

\cventry{}{Google Scholar}{\url{https://scholar.google.com/citations?user=pAfLhMoAAAAJ}}{}{}{}{}
\cventry{}{ORCID}{\url{https://orcid.org/0000-0001-7248-4014}}{}{}{}{}

\end{document}

\typeout{ *************** End of file cv *************** }
