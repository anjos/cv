\typeout{ ====================================================================}
\typeout{ This is file snsf-achievements, created at Mon 14 May 11:45:34 2018 CEST }
\typeout{ Andre Anjos <andre.anjos@idiap.ch>                                  }
\typeout{ ====================================================================}

\documentclass[a4paper,10pt,onecolumn]{article}
\usepackage[T1]{fontenc}

% Choose your font here - combinations with harmonized math symbols suggested
%\usepackage{lmodern} %Times-like
\usepackage{newpxtext} \usepackage[euler-digits]{eulervm} %Palatino-like
%\usepackage{libertine} \usepackage[libertine]{newtxmath} %Libertine-like

% sorting=none will preserve the bib file order
\usepackage[style=verbose-ibid,backend=biber,sorting=none,maxbibnames=3]{biblatex}
\addbibresource{publications.bib}

% adjust the page margins
\usepackage[scale=0.82]{geometry}

\usepackage{hyperref}
\usepackage{titling}

\pretitle{\vspace{-6em}\begin{flushleft}\LARGE}
\posttitle{\par\end{flushleft}}
\preauthor{}
\postauthor{}
\author{}
\predate{}
\postdate{}
\date{}

\hypersetup{
backref,%
colorlinks,%
citecolor=black,%
filecolor=black,%
linkcolor=black,%
urlcolor=black
}


\begin{document}

% List up to three major scientific achievements in the career of each
% applicant and briefly describe the individual applicant's specific
% contribution and the impact of his/her work. Each list should not exceed one
% DIN A4 page.

\title{Scientific Achievements (Dr. André Anjos)}
\maketitle

\section{Semantic Segmentation for Medical Imaging}

Since the introduction of U-Nets in 2015, the field of medical image
segmentation has seen renewed interest bringing in a variety of fully
convolutional (deep) neural network (FCN) architectures for binary and
multi-class segmentation problems promising very attractive results, with
applications in computed tomography, retinography, and histopathology to cite a
few.  Despite the incredible progress, the lack of annotated images (due to
cost), and rigor in the comparison of trained models has led the community to
believe larger and more dense networks provide better results.

I tried to address these gaps in two different ways.  The
first\footcite{arxiv-2019} was to conduct and publish rigorous (open source,
reproducible) benchmarks with popular retinography datasets and
state-of-the-art FCN models in which we: i) showed that simple transformation
techniques can be a surprisingly effective way to improve segmentation quality
in unseen conditions; ii) we proposed a set of plots and metrics that give
additional insights into model performance and demonstrated via tables and
plots how to take advantage of that information, throwing a new light over some
published benchmarks\footnote{\url{https://gitlab.idiap.ch/bob/bob.ip.binseg}}.
In a second contribution\footcite{arxiv-2020} we propose that a minimalistic
version of a standard U-Net with 3 orders of magnitude less parameters,
carefully trained and rigorously evaluated, closely approximates the
state-of-the-art performance in vessel segmentation for retinography.

\section{Contributions to Computer Vision and Deep Learning in Biometrics}

I have actively worked in computer vision and deep learning (mostly) associated
to biometric recognition, with potential application to various other tasks.
Contributions range from the collection of datasets, the exploration of
different methods to address and assess biometric recognition vulnerabilities,
domain adaptation, and remote photoplethysmography.  I highlight here two
contributions that are related to our proposal.  Domain Specific Units
(Adaptation)~\footcite{tifs-2019}: we apply domain adaptation via dedicated
Domain-Specific Units (DSU), with an application to Heterogeneous Face
Recognition.  We developed a mechanism to adapt the parameters of models
pre-trained on large visual spectral face recognition datasets, which are
readily available.  My contributions are directly related to core idea of this
work.  Remote Photoplethysmography (rPPG):  This study\footcite{arxiv-2017-2}
tackles the problem of reproducibility in rPPG.  We introduded a new, publicly
available dataset containing a relatively large number of subjects recorded
under two different lighting conditions.  We then thoroughly benchmarked
state-of-the-art algorithms using an unbiased experimental evaluation in
various settings.  We showed that none of the selected algorithms is precise
enough to be used in a real-world scenario.

\section{Reproducible Research}

We've been actively looking at the reproducibility of published work and how to
lower the entrance barrier of publication readers, converting them into engaged
users of methods we create.  We argue it is insufficient, in most cases, to
only publish software leading to results if original data remains inaccessible.
In particular\footcite{icml-2017-1}, we note that reproducibility
should imply in the following characteristics: repeatability, shareability,
extensibility and stability, which is not guaranteed by most published material
to date.  We proposed an open-source software suite called Bob
(\url{https://www.idiap.ch/software/bob}) that possesses such characteristics,
demonstrating its flexibility to various tasks including Medical Image
Segmentation (\url{https://gitlab.idiap.ch/bob/bob.ip.binseg}), Remote
Photoplethysmography (\url{https://gitlab.idiap.ch/bob/bob.rppg.base}), and
Biometric Person Recognition (\url{https://gitlab.idiap.ch/bob/bob.bio.base}).

\end{document}

\typeout{ *************** End of file cv *************** }
